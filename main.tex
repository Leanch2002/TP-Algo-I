\documentclass[11pt,a4paper]{article}
\usepackage[utf8]{inputenc}
\usepackage{graphicx}
\usepackage[left=2.5cm,top=3cm,right=2.5cm,bottom=3cm,bindingoffset=0.5cm]{geometry}
\usepackage{AEDLogica, AEDEspecificacion, AEDTADs}
\usepackage{caratula}


\titulo{Trabajo práctico}
\subtitulo{Especificacion de TADs}

\fecha{\today}

\materia{Algoritmos y Estructuras de Datos}
\grupo{BobElConstructorPorCopia}

\integrante{Choque, Leandro}{252/25}{leandroch2002@gmail.com}
\integrante{Musi, Santino}{965/24}{santinomusi1@gmail.com}
\integrante{Rojas, Damian}{209/25}{dam.rojas1@gmail.com}
\integrante{Martell, Juan Bautista}{622/25}{Juanbamartell@hotmail.com}
% \integrante{Apellido, Nombre2}{002/01}{email2@dominio.com} %


% Declaramos donde van a estar las figuras
% No es obligatorio, pero suele ser comodo
\graphicspath{{../static/}}

% Asi pueden escribir nuevos comandos. 
% Este por ejemplo asegura q los nombres 
% que figuren con una tipografia diferenciada  
\newcommand{\Tipo}[1]{\mathsf{#1}}
% la sintaxis es \newcommand{\nombreDeLaMacro}[cantidadDeParametros]{Lo que va ser remplazado por el macro} 
\newcommand{\norm}[1]{\vert #1\vert}

% Empieza el documento
\begin{document}

\maketitle

\section{Supongo que acá iría una descripción}

Breve descripción.

Luego veremos bien el formato, esto de momento es para tener un esqueleto.

\section{Especificacion}

% Acá arrancamos el TAD --------------------------------------------------
\begin{tad}{EdR}

% SECCIÓN DE DUDAS/COMENTARIOS --------------------------------------------------

% Preguntar si estudiantes necesita nombre o identificación
% Preguntar por \yLuego
% Preguntar por igualdad(), si basta con '=' en arrays

% RENOMBRE DE TIPOS --------------------------------------------------

% Aula: Array de Array de Examen
% Estudiante: -
% Nota: Int
% Examen: Seq(Seq(Char)) # Me parece que esto sería Seq(Char)

% es decir, cada "instancia" de Examen, sería un array con las respuestas
% ej: examen1 = ['A', 'F', 'J', 'C'], uno sin responder: ['', '', '', '']
% y uno a medio responder: ['B', '', '', 'D'] 
% (El largo lo determina la cantidad de ejercicios)

\obs{aula}{Aula}
\\
\obs{solucion}{Examen}
\\

% Metodo EdR --------------------------------------------------
\begin{proc}{EdR}
{
\In a: Aula,
\In s: Examen,
\In cantEstudiantes: \Z
}{
\Tipo{EdR}
}
    \requiereLargo{aulaCuadrada(a) \yLuego \\ examenValido(solucion) \yLuego \\ cantValidaEstudiantes(a, cantEstudiantes) }
    \aseguraLargo{
        (\norm{res.aula} = \norm{a}) \yLuego
    \\ (\paraTodo{i}{\Z}{0 \leq i < \norm{a} \implicaLuego \norm{res.aula[i]} = \norm{a[i]}}) \yLuego
    \\ (\paraTodo{j}{\Z}{0 \leq j < \norm{res.aula} \implicaLuego noHayAlumnosJuntos(res.aula[j])}) \yLuego
    \\ (cuantosEstudiantes(res.aula) = cantEstudiantes) \yLuego
    \\ (res.solucion = s)
    }
\end{proc}

\predLargo{aulaCuadrada}{a: Aula}{
    \paraTodo{i}{\Z}{0 \leq i < \norm{a} \implicaLuego \norm{a[i]} = \norm{a}}
}

\predLargo{examenValido}{s: Examen}{
    (True) % Obviamente hay que completar esto XD
}

\predLargo{cantValidaEstudiantes}{a: Aula, e: \Z}{
    ifthenelse(esPar(\norm{a}^ 2), \frac{\norm{a}^ 2}{2}, \frac{\norm{a}^ 2 + 1}{2})   
}

% Cree el predicado, lo adapté a la modularización y le cambié algo que era redundante.
% Fijense que tuve que pones que fila es del tipo Array de Examen, podríamos hacer que sea tipo Fila?
\predLargo{noHayAlumnosJuntos}{fila: \seq{Examen}}{
    \paraTodo{i}{\Z}{0 <= i < \norm{fila}-1 \implicaLuego (hayAlumno(fila[i]) \implica \neg(hayAlumno(fila[i + 1])))}
}

% Así estaba antes, lo dejo por sí no estan de acuerdo con el cambio XD
% \paraTodo{j}{\Z}{0 < j < \norm{aula[i]}-1 \implicaLuego (hayAlumno(aula[i][j]) \implica \neg (hayAlumno(aula[i][j-1]) \lor hayAlumno(aula[i][j+1])))}

\predLargo{cuantosEstudiantes}{a: Aula}{
    (True) % Obviamente hay que completar esto XD
}

% Metodo igualdad --------------------------------------------------
\begin{proc}{igualdad}
{
\In edr1,edr2: EdR,
}{
\Tipo{Bool}
}
    \requiere{True}
    \aseguraLargo{(res = True) \leftrightarrow
    \\ (edr1.aula = edr2.aula) \yLuego
    \\ (edr1.solucion = edr2.solucion)
    }
\end{proc}

% Metodo copiarse --------------------------------------------------
\begin{proc}{copiarse}
{
\In alumno: Estudiante % que pase las coordenadas o su examen? 
}{
\Tipo{EdR}
}
    \requiere{True} % Que el alumno exista (Vemos cómo ponerlo depende de cómo identificamos al alumno) y que tenga a alguien en frente o a dos asientos de distancia a los lados
    \asegura{res} % aula[i][j].examen[4] (para la pregunta 4 por ejemplo) == aula[i+1][j].examen[4] o aula[i][j+2] o aula[i][j+3] o aula[i][j-2] o aula[i][j-3] (Chequear si existen esas coordenadas)
\end{proc}

% Metodo publicarResolucion --------------------------------------------------
\begin{proc}{publicarResolucion}
{
\In completar, Completar
}{
\Tipo{EdR}
}
    \requiere{True}
    \asegura{res}
\end{proc}

% Metodo resolver --------------------------------------------------
\begin{proc}{resolver}
{
\In completar, Completar
}{
\Tipo{EdR}
}
    \requiere{True}
    \asegura{res}
\end{proc}

% Metodo entregar --------------------------------------------------
\begin{proc}{entregar}
{
\In alumno: Estudiante
}{
\Tipo{EdR}
}
    \requiere{True} % Que el alumno exista
    \asegura{res} % que aula[i][j] pase a estar vacío? Es suficiente?
\end{proc}

% Metodo chequearCopias --------------------------------------------------
\begin{proc}{chequearCopias}
{
\In completar, Completar % Creo que no recibe nada o recibe un EdR
}{
\Tipo{seq<Estudiante>}
}
    \requiere{True} % Si no recibe nada va True, a lo mejor recibe un EdR pero entonces también iría True, mientras sea un EdR válido sirve
    \asegura{res} % Acá hay que ver cómo poner lo de propoción de rtas equivalentes a compañero cercano >60% y eso, cómo medirlo, o eso sería muy implementativo?
    % Quedaría tipo, para todo alumno en aula, si esSimilarACercano(aula[i][j].examen) o esSimilarAAlumnado(aula[i][j].examen) entonces se concatena a la secuencia
\end{proc}

% Metodo corregir --------------------------------------------------
\begin{proc}{corregir} 
{ %Acá dice que tenemos que corregir los examenes de los que no fueron sospechosos de copiarse, pero no podemos llamar un proc adentro de otro, o no?
\In completar, Completar
}{
\Tipo{seq<<estudiante, nota>>}
}
    \requiere{True}
    \asegura{res}
\end{proc}

\end{tad}

\end{document}